\textbf{Las ventajas:}

\begin{itemize}
    \item \textbf{Entorno Seguro de Aprendizaje:} Los usuarios pueden aprender, construir y programar robots en un entorno simulado sin riesgo de dañar equipos o de poner en peligro la seguridad de los estudiantes.
    
    \item \textbf{Variedad de Herramientas Educativas:} Ofrece funcionalidades como la construcción de robots, simulación de competencias, y diseño de entornos de práctica, adaptados a diferentes niveles de habilidad y objetivos de aprendizaje.
    
    \item \textbf{Flexibilidad en el Entrenamiento:} Los usuarios pueden personalizar los entornos de simulación y adaptar sus entrenamientos a diferentes temas o tipos de competencia, lo cual es ideal para quienes desean un aprendizaje autodirigido o enfocado en competencias específicas.
\end{itemize}

\textbf{Desventajas:}

\begin{itemize}
    \item \textbf{Curva de Aprendizaje para Funcionalidades Avanzadas:} Aunque es fácil para principiantes, algunas características avanzadas pueden requerir un tiempo de adaptación, lo cual puede ser un desafío para quienes no tienen conocimientos previos en robótica.
    
    \item \textbf{Dependencia de Recursos Computacionales:} La simulación de efectos físicos y gráficos puede requerir una computadora con buenos recursos (procesador y tarjeta gráfica), lo cual puede limitar el acceso en entornos con computadoras de bajo rendimiento.
    
    \item \textbf{Necesidad de pagar planes:} Esto se presenta como desventaja ya que para hacer uso de algunos recursos de construcción, es necesario pagar algunos de los planes, también el acceso a clases.
\end{itemize}
