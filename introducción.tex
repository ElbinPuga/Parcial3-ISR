La simulación se está convirtiendo rápidamente en una parte integral del flujo de trabajo de diseño robótico. Con demasiada frecuencia, no es factible probar con robots de hardware tanto nuevos diseños físicos como software de control. Las computadoras son fundamentales para producir rápidamente nuevos diseños y probar sus capacidades antes de construir cualquier hardware. Los robots autónomos requieren una gran cantidad de capacidades algorítmicas para permanecer seguros y completar tareas. Limitar las pruebas al hardware restringe severamente las posibilidades y, al mismo tiempo, aumenta el riesgo para el hardware ~\cite{gucwa cheng}.
Los robots educativos se están convirtiendo en un método cada vez más popular para implementar actividades prácticas en las aulas. Lego Mindstorms es el kit de robótica para el aula más utilizado, pero no está diseñado para integrarse directamente en las lecciones de matemáticas. La cantidad de piezas y la reconfigurabilidad son excelentes para estimular el pensamiento creativo, pero no favorecen el aprendizaje en aulas grandes. Los robots simples que transmiten los conceptos educativos permiten a los estudiantes aprender matemáticas en primer lugar y sobre todo con la ayuda de la robótica ~\cite{gucwa cheng}.
