\textbf{Educación:} Es utilizado en instituciones educativas, como escuelas y universidades, para enseñar robótica, programación y fundamentos de ingeniería. RoboSim permite a los estudiantes experimentar con robots virtuales, desarrollar habilidades prácticas sin necesidad de equipos físicos y comprender conceptos de programación de manera visual e interactiva.

\textbf{Entrenamiento en Competencias de Robótica:} RoboSim es una herramienta ideal para entrenarse en competencias de robótica. Permite a los usuarios participar en desafíos de programación y diseño de robots en línea, preparándolos para competiciones reales sin requerir hardware costoso.

\textbf{Centros de Investigación y Desarrollo:} Aunque RoboSim está orientado a la educación, también puede utilizarse en laboratorios o centros de investigación para prototipado rápido. Los investigadores pueden probar algoritmos de control y modelos de simulación en un entorno virtual antes de aplicarlos a robots físicos.

\textbf{Industria y Capacitación Técnica:} Empresas que implementan robótica en sus operaciones pueden usar RoboSim para capacitar a sus empleados. Esto resulta útil para comprender el funcionamiento y programación de robots industriales en un entorno seguro, permitiendo aprender y resolver problemas de programación y diseño sin riesgos.

\textbf{Desarrollo Personal y Aficionados:} RoboSim también es accesible para aficionados y personas que desean aprender robótica de manera autodidacta. Ofrece un entorno controlado y de bajo costo para experimentar, aprender y desarrollar habilidades en programación de robots.
